% $Id: article.tex 139 2015-06-26 06:06:15Z tjn3 $

\documentclass{article}

% TJN: Must define before includes as fancyheader uses '\draftmark' macro
\newcommand{\draftmark}{}

\include{defs}
\include{macros}


\title{Thesis proposal}
\author{Zachary Parchman}
\date{}

\begin{document}
\maketitle

\thispagestyle{empty}

\section{Introduction}
This thesis proposal focuses on the creation of an online application
specific checkpoint restart mechanisms. 
These mechanisms will allow an application to recover the data that belongs to 
failed nodes so the application can continue to make progress without 
restarting. 
To achieve this efficiently the application will need to be able to
perform a trade-off between the resilience of checkpoints and the cost of
checkpoints so that both highly 
reliable checkpoints and lower cost checkpoints can be integrated into the
application. This will allow it to suffer process failures while continuing
to make
progress toward completion.

\section{Background}
\subsection{Message Passing Library}
Message Passing Library (MPI) is a library intended for use in distrusted
memory computing. The usual use case is for the application to
consist of one executable that is ran on more than one machine. These
executables use their process number, rank in MPI, to determine what actions
to take. Communication is achieved through a rich set of message that the
processes can send to each other. These messages are translated into
efficient operations taking advantage varied of communication hardware so
that the application does not have to concern itself with the underlying
communication hardware. 

While MPI contains a rich set of communication functions it contains no
fault tolerance. By default when an error occurs the entire application,
every process in the job, is aborted. This can be changed with an error
handler function, but after a failure occurs the MPI library can no longer
be used for communication, even with a user defined error handler.

\subsection{ULFM}
User Level Fault Mitigation (ULFM)\cite{ulfm-site} is proposed as the
technology to use with MPI to make applications fault tolerant. It is a
proposed extension to the MPI library that allows for a well written program
to take corrective action when a process fails. ULFM does not provide any
facilities for data checkpointing. This differs from current popular
methods of adding fault tolerance into an MPI program in that the program
must be written specifically to use ULFM. ULFM does not itself provide fault
tolerance. ULFM instead provides fault detection and the ability to bring
the MPI library back into a state where communication can take place. 

While ULFM allows a program to continue to use MPI after a failure; it does
not provide a way for an application to retrieve it's old data. There is no
checkpointing system in ULFM. This forces the application to perform it's
own data replication if it wants to continue making progress after a process
failure.

\section{Failure Mode Supported}
As ULFM only supports total process failures the proposed checkpointing
system that will be built on top of it will also only support total process
failures. A total process failure is defined as having the failing process
cease all communication without warning. This is assumed to be permanent for
the life of the application.


\section{Contribution}
I shall design, implement, and evaluate an online application specific
checkpoint restart mechanism intended to preserve the data from failed ranks
in ULFM
applications. The checkpoint system will be able to support multiple
checkpointing strategies with differing costs in latency, bandwidth, and
memory usage.
The contribution of this work shall be a library to provide applications the
ability to create runtime checkpoints that have not only variable cost, but
also do not require restarting the application. 
The application will be able to contain multiple independent subsets of its
processes that can fail and recover without communication between the
subsets during processing.  More over I will show that there is a
efficiency driven reason to desire this over a more traditional checkpoint
restart mechanisms. 

\section{Evaluating}
To evaluate the library I intend to use the NAS parallel benchmarks IS
benchmark. This benchmark sorts an array of integers in distributed memory.
Mantevo's HPCCG benchmark, a short conjugate gradient application, will also
be used. A competing application specific checkpoint restart system will be
chosen. Currently under consideration is libckpt.
The evaluation will focus on checkpoint latency, the time taken to
recover from a failure, and runtime penalty for using the checkpoint restart
system when compared to when no resiliency is used. 
These metrics will be determined by performing multiple runs with
varying failure situations on a distributed memory machine.

% XXX: cite everything
\nocite{*}

%-------------------------------------------------------------------------
%   Bibliography
%-------------------------------------------------------------------------
\bibliographystyle{plain}
\bibliography{bibs/CITES}

\end{document}

% vim:tabstop=4:shiftwidth=4:expandtab:textwidth=76:filetype=plaintex
