% $Id: article.tex 139 2015-06-26 06:06:15Z tjn3 $

\documentclass{article}

% TJN: Must define before includes as fancyheader uses '\draftmark' macro
\newcommand{\draftmark}{}

% "defs.tex"

\pagestyle{empty}
\usepackage{times}
\usepackage{graphicx} 
%\usepackage{latex8}
\usepackage{subfigure,epsfig}
\usepackage{caption}
\usepackage{fullpage}

% Force paper size (not A4)
\ExecuteOptions{letterpaper}

% So we can IF-Def our LaTeX papers too!
\usepackage{ifpdf}

% Needed for compact list (i.e., compactitemize, etc. stuff)
\usepackage{mdwlist}

% Needed for alt. compact list (i.e., compactenum, compactitem, etc.)
% (Used in tables b/c it appears to work better than mdwlist.)
\usepackage{paralist}

%for setting indentation in lists
\usepackage{enumitem}

\usepackage{algorithm}
\usepackage{algpseudocode}
%\usepackage{algorithmic}
\usepackage{algorithmicx}

\usepackage{framed, color}

\usepackage[final]{listings}
\definecolor{mygray}{rgb}{0.5,0.5,0.5}

\lstdefinestyle{my-c-numbered}{
  language=C,
  showspaces=false,
  columns=fullflexible,
  breaklines=true,
  frame=single,
  emphstyle=\bf,
  basicstyle=\footnotesize,
  stringstyle=\ttfamily,
  breaklines=true,
  numbers=left,
  numberstyle=\tiny,
  stepnumber=1
}

\lstdefinestyle{zlua}{
  language=[5.2]Lua,
  % TJN: If have problems with Lua language for listings package, comment
  % above and uncomment the following line to use 'C' as language type.
  %language=C,
  showspaces=false,
  columns=fullflexible,
  breaklines=true,
  frame=single,
  commentstyle=\rmfamily,
  emphstyle=\bf,
  basicstyle=\footnotesize,
  stringstyle=\ttfamily,
  breaklines=true,
  numbers=left,
  numberstyle=\tiny,
  stepnumber=1
}

\lstdefinestyle{my-bourne-numbered}{
  language=bash,
  showspaces=false,
  columns=fullflexible,
  breaklines=true,
  frame=single,
  emphstyle=\bf,
  basicstyle=\footnotesize\ttfamily\scriptsize,
  commentstyle=\itshape\color{mygray},
  stringstyle=\ttfamily,
  breaklines=true,
  numbers=left,
  numberstyle=\tiny,
  stepnumber=1
}

\lstdefinestyle{my-docker-numbered}{
  language=bash,
  showspaces=false,
  columns=fullflexible,
  breaklines=true,
  frame=single,
  emphstyle=\bf,
  basicstyle=\footnotesize\ttfamily\scriptsize,
  commentstyle=\itshape\color{mygray},
  stringstyle=\ttfamily,
  breaklines=true,
  numbers=left,
  numberstyle=\tiny,
  stepnumber=1
}

\lstdefinestyle{my-xml-numbered}{
    language=XML,
    showspaces=false,
    numbers=left,
    numberstyle=\tiny,
    basicstyle=\footnotesize\ttfamily\scriptsize,
    stringstyle=\ttfamily,
    showstringspaces=false,
    keywordstyle=\scriptsize,
    emphstyle={\color{Cyan}},
    morecomment=[s]{<!--}{-->},
    commentstyle=\itshape\color{mygray},
}


%
% PDFpages options
%  \usepackage[<opts>]{pdfpages}
%           final: inserts pages (default)
%           draft: does not insert pages, prints box & the filename instead
%   enable-survey: activates survey functionality (experimental)
%
% Usage:
%   \includepdf[<key=val>]{<filename>}
%        <key=val>: A comma seperated list of options (see package docs)
%       <filename>: Filename of the PDF doc (no blanks!)
%
% See also:
%   http://www.ctan.org/tex-archive/macros/latex/contrib/pdfpages/pdfpages.pdf
%
\usepackage[final]{pdfpages}
%\usepackage[draft]{pdfpages}


%
% Appendix package
%
\usepackage{appendix}



%
%-------PDF stuff--------
%
% see http://ringlord.com/publications/latex-pdf-howto/
% or [Kopka99book, p394-397]
\newif\ifpdf
\ifx\pdfoutput\undefined
	\pdffalse       % we are not running pdflatex
\else
	\pdfoutput=1    % we are running pdflatex
	\pdftrue
\fi
\ifpdf
% You can set the title, author etc. as needed...
\usepackage[pdftex,
	pdftitle={Fault Tolerance and System-level Virtualization in Large-Scale
    Computing},
	pdfauthor={Thomas J. Naughton, III},
	pdfsubject={},
	pdfpagemode={UseOutlines},  
	bookmarks,bookmarksopen,
	pdfstartview={FitH},
    % TJN: To make PDF hyperlinks in paper black, comment/uncomment 'linkcolor'
    % colorlinks,linkcolor={black},citecolor={black},
	colorlinks,linkcolor={blue},citecolor={blue},
	urlcolor={red},
]{hyperref}
\fi
%-------PDF stuff--------



%
%-------DRAFT stuff--------
%
% 'Draft' watermarks
%
% Using options, to lighten the watermark, and only do it on select page/area
%\usepackage[light,outline,first,bottom]{draftcopy}
%\usepackage[light,outline,first]{draftcopy}
%
% TJN: pdfdraftcopy seems to work with both PS & PDF, but w/ pdflatex issues
%\usepackage{pdfdraftcopy}
% so I typically just include the following & ps2pdf the file.
%\usepackage{draftcopy}
%
% - - - - - - - - - - - - - - - - 
% *** Yet Another Method ***
%
% Look at the 'svn-multi' (aka svnkw) page from CTAN, which allows
% for typesetting control over SVN revision information on a per-file
% and/or whole document level.
% See also:
%   - The "LaTeX & Subversion" link/info in the Trac wiki for "pubs"
%      https://bear.csm.ornl.gov/trac/pubs/
%
%  \usepackage{svn}
%  \usepackage{svn-multi}
%    ...
%  \svnid{$Id: defs.tex 3825 2015-06-26 05:44:36Z tjn3 $}
%    ...
%  Paper Revision:~\svnrev \\
%

\usepackage{svn}
%\usepackage{svn-multi}
\usepackage{svnkw}

%-------DRAFT stuff--------


%
% This gives the heading the funny little footnote markers...then switch to 
% standard arabic as shown below for main body of article.  
% Ref: [p.110-111]{Kopka99book}
\renewcommand{\thefootnote}{\fnsymbol{footnote}}




%-----------------------------------------
% Header / Footer stuffo
%
% NOTE: Using the '\draftmark' macro, which must
%       be defined prior to use of this code.
%       This is typically done in the top-level
%       LaTeX document that is including this 
%       'defs.tex' file.

\usepackage{fancyhdr}
  \pagestyle{fancy}
  \fancyhf{}                             % Clear all other settings
  \renewcommand{\headrulewidth}{0.0pt}   % get rid of the rule-line
  \lhead{}
  \chead{} 
  \rhead{}
  \lfoot{}
  \cfoot{}
  \lfoot{\begin{small}\emph{\draftmark}\end{small}}

%-----------------------------------------



%%%%
% (TJN: Used for \xxxtodo and \xxxtodoMargin macros)
%
% textwidth sets width of todo items in margin
\usepackage[colorinlistoftodos, textwidth=2.5cm, shadow]{todonotes}
%
% Uncomment to DISABLE all todonotes
%\usepackage[disable, colorinlistoftodos, textwidth=2.5cm, shadow]{todonotes}
%%%%


%%%%%%%%%%%%%%%%%%%%%%%%%%%%%%%%%%%%%%%%%
% LaTeX Macros
%%%%%%%%%%%%%%%%%%%%%%%%%%%%%%%%%%%%%%%%%


%
% Setup a few helpful formatting macros
%


% Use urlstyle to protect for usage with functions containing underscores
\newcommand\func{\begingroup \urlstyle{tt}\Url}
\newcommand\variable{\begingroup \urlstyle{tt}\Url}
\newcommand\envvar{\begingroup \urlstyle{tt}\Url}
\newcommand\file{\begingroup \urlstyle{tt}\Url}

\newcommand{\namespace}[1]{\texttt{#1}}

%\newcommand{\variable}[1]{\texttt{#1}}
% \newcommand{\file}[1]{\texttt{#1}}
\newcommand{\cmd}[1]{\texttt{#1}}
\newcommand{\function}[1]{\texttt{#1}}
\newcommand{\manpage}[2]{\texttt{#1(#2)}}
\newcommand{\user}[1]{\texttt{#1}}
\newcommand{\hostname}[1]{\texttt{#1}}


 % TJN: Mirror the acronym macros with the Glossary macros
 %      so it is earier to use in the manner I tend to use the package.
 % This forces the "full" verison of glossary, i.e., first use string,
 % Example: \glsfull{ft} => "fault tolerance~(FT)".
\newcommand{\glsfull}[1]{\glsreset{#1}\gls{#1}}

%-----
% Changebar macro
%  Usage:
%     \changebegin{Thomas} ...my modified lines...  \changeend{}

\newcommand{\changebegin}[1]{\marginpar[\hspace*{-60pt}\mbox{\hspace*{10pt} 
 $\top$ \tiny ({#1})}]{\mbox{$\top$ \tiny ({#1})}}}
\newcommand{\changeend}[1]{\marginpar[\hspace*{-60pt}\mbox{\hspace*{10pt} 
 $\bot$ \tiny ({#1})}]{\mbox{$\bot$ \tiny ({#1})}}}
%
% To get rid of these change marks (without actually removing from text)
%\newcommand{\changebegin}[1]{} 
%\newcommand{\changeend}[1]{} 
%-----



%-----
%
% Discussion macro
%  Usage:
%     \beging{discuss} ...discussion/comments...  \end{discuss}

\newenvironment{discuss}{\begin{small}\begin{list}{}{}\item[]{\it \underline{\textcolor{cyan}{Discussion item:}}}
\addcontentsline{toc}{subsection}{\textcolor{cyan}{Discussion Item}}}{{\rm ({\it \textcolor{cyan}{End of discussion item.}})} \end{list}\end{small}}
%
%-----


%-----
%
% Review Text macro -- place this around text that needs 
%                      to be reviewed/revised. 
%                      (Note: Not sure how macro will work across sections.)
% Usage:
%   \begin{reviewme}
%     ...some text in the doc to review...
%   \end{reviewme}
%

\newenvironment{reviewme}{\begin{paragraph}{\textcolor{magenta}{\it Review text:}}
\addcontentsline{toc}{subsection}{\textcolor{magenta}{Review text:}}
\changebegin{\textcolor{magenta}{ReviewME}}}
{{\rm ({\textcolor{magenta}{\it End of review text.}})\changeend{\textcolor{magenta}{ReviewME}}}\end{paragraph}}
%
% Alt. version: much more compact (doesn't show Begin/End text marks).
%
%\newenvironment{reviewme}{\addcontentsline{toc}{subsection}{\textcolor{magenta}{Review text:}}
%\changebegin{\textcolor{magenta}{ReviewME}}}{\changeend{\textcolor{magenta}{ReviewME}}}
%
%-----


%%%%%%%%%%%%%%%%%%%%%%
% Create compact lists
%%%%%%%%%%%%%%%%%%%%%%
% Usage: 
%
%    [DEFAULT]                       [ALT-VERSION]
%   \begin{itemize*}               \begin{compactitemize}
%      \item xxx       -OR-           \item xxx
%   \end{itemize*}                 \end{compactitemize}
%
% Add alternate version with more descriptive LaTeX keywords,
% the package provides the "...*" (star) version by default.
\makecompactlist{compactenumerate}{enumerate}
\makecompactlist{compactitemize}{itemize}
\makecompactlist{compactdescription}{description}


%-----
%
% Attention/Notice macro
%  Usage:
%     \beging{AttentionNote} ...note/comments...  \end{AttentionNote}
%
\newenvironment{AttentionNote}{\begin{quote}\begin{list}{}{}\item[] 
    \hfil\rule{0.8\textwidth}{.4pt}\hfil \\
    {\bf Notice:\\}}{{
    \\ \hfil\rule{0.8\textwidth}{.4pt}\hfil \\
    } \end{list}\end{quote}}
%
%-----

%-----
%
% ToDo macro
%   Usage:
%      \xxxtodo{INITIALS}{Todo text here}
%      \xxxtodoMargin{INITIALS}{Todo text for margin here}
%
%   Example
%      \xxxtodo{TJN}{This would be a good example for todo.}
%      \xxxtodoMargin{TJN}{And something in the margin}
%
%   Descr:
%      Places a numbered 'todo' with the given initials and text into
%      the document with a MS-Word style yellow comment box.
%      The "Margin" version, simply puts the todo box in the margin,
%      which can be helpful for general comments to a section/paragraph.
%      
%%
\newcounter{todocounter}
\newcommand{\todonum}[2][]
{\stepcounter{todocounter}\todo[#1]{\thetodocounter: #2}}
%
% ToDo comments
\newcommand{\xxxtodo}[2]{\todonum[color=yellow!30,inline,size=\footnotesize,caption=Comment~\thetodocounter]{#1: #2}}
\newcommand{\xxxtodoMargin}[2]{\todonum[color=yellow!30,size=\scriptsize,noline,caption=MarginComment~\thetodocounter]{#1: #2}}
%
%-----





\title{Thesis proposal}
\author{Zachary Parchman}
\date{}

\begin{document}
\maketitle

\thispagestyle{empty}

\section{Introduction}
This thesis proposal focuses on the creation of an online application
specific checkpoint restart mechanisms. 
These mechanisms will allow an application to recover the data that belongs to 
failed nodes so the application can continue to make progress without 
restarting. 
To achieve this efficiently the application will need to be able to
perform a trade-off between the resilience of checkpoints and the cost of
checkpoints so that both highly 
reliable checkpoints and lower cost checkpoints can be integrated into the
application. This will allow it to suffer process failures while continuing
to make
progress toward completion.

\section{Background}
\subsection{Message Passing Library}
Message Passing Library (MPI) is a library intended for use in distrusted
memory computing. The usual use case is for the application to
consist of one executable that is ran on more than one machine. These
executables use their process number, rank in MPI, to determine what actions
to take. Communication is achieved through a rich set of message that the
processes can send to each other. These messages are translated into
efficient operations taking advantage varied of communication hardware so
that the application does not have to concern itself with the underlying
communication hardware. 

While MPI contains a rich set of communication functions it contains no
fault tolerance. By default when an error occurs the entire application,
every process in the job, is aborted. This can be changed with an error
handler function, but after a failure occurs the MPI library can no longer
be used for communication, even with a user defined error handler.

\subsection{ULFM}
User Level Fault Mitigation (ULFM)\cite{ulfm-site} is proposed as the
technology to use with MPI to make applications fault tolerant. It is a
proposed extension to the MPI library that allows for a well written program
to take corrective action when a process fails. ULFM does not provide any
facilities for data checkpointing. This differs from current popular
methods of adding fault tolerance into an MPI program in that the program
must be written specifically to use ULFM. ULFM does not itself provide fault
tolerance. ULFM instead provides fault detection and the ability to bring
the MPI library back into a state where communication can take place. 

While ULFM allows a program to continue to use MPI after a failure; it does
not provide a way for an application to retrieve it's old data. There is no
checkpointing system in ULFM. This forces the application to perform it's
own data replication if it wants to continue making progress after a process
failure.

\section{Failure Mode Supported}
As ULFM only supports total process failures the proposed checkpointing
system that will be built on top of it will also only support total process
failures. A total process failure is defined as having the failing process
cease all communication without warning. This is assumed to be permanent for
the life of the application.


\section{Contribution}
I shall design, implement, and evaluate an online application specific
checkpoint restart mechanism intended to preserve the data from failed ranks
in ULFM
applications. The checkpoint system will be able to support multiple
checkpointing strategies with differing costs in latency, bandwidth, and
memory usage.
The contribution of this work shall be a library to provide applications the
ability to create runtime checkpoints that have not only variable cost, but
also do not require restarting the application. 
The application will be able to contain multiple independent subsets of its
processes that can fail and recover without communication between the
subsets during processing.  More over I will show that there is a
efficiency driven reason to desire this over a more traditional checkpoint
restart mechanisms. 

\section{Evaluating}
To evaluate the library I intend to use the NAS parallel benchmarks IS
benchmark. This benchmark sorts an array of integers in distributed memory.
Mantevo's HPCCG benchmark, a short conjugate gradient application, will also
be used. A competing application specific checkpoint restart system will be
chosen. Currently under consideration is libckpt.
The evaluation will focus on checkpoint latency, the time taken to
recover from a failure, and runtime penalty for using the checkpoint restart
system when compared to when no resiliency is used. 
These metrics will be determined by performing multiple runs with
varying failure situations on a distributed memory machine.

% XXX: cite everything
\nocite{*}

%-------------------------------------------------------------------------
%   Bibliography
%-------------------------------------------------------------------------
\bibliographystyle{plain}
\bibliography{bibs/CITES}

\end{document}

% vim:tabstop=4:shiftwidth=4:expandtab:textwidth=76:filetype=plaintex
