% "defs.tex"

\pagestyle{empty}
\usepackage{times}
\usepackage{graphicx} 
%\usepackage{latex8}
\usepackage{subfigure,epsfig}
\usepackage{caption}
\usepackage{fullpage}

% Force paper size (not A4)
\ExecuteOptions{letterpaper}

% So we can IF-Def our LaTeX papers too!
\usepackage{ifpdf}

% Needed for compact list (i.e., compactitemize, etc. stuff)
\usepackage{mdwlist}

% Needed for alt. compact list (i.e., compactenum, compactitem, etc.)
% (Used in tables b/c it appears to work better than mdwlist.)
\usepackage{paralist}

%for setting indentation in lists
\usepackage{enumitem}

\usepackage{algorithm}
\usepackage{algpseudocode}
%\usepackage{algorithmic}
\usepackage{algorithmicx}

\usepackage{framed, color}

\usepackage[final]{listings}
\definecolor{mygray}{rgb}{0.5,0.5,0.5}

\lstdefinestyle{my-c-numbered}{
  language=C,
  showspaces=false,
  columns=fullflexible,
  breaklines=true,
  frame=single,
  emphstyle=\bf,
  basicstyle=\footnotesize,
  stringstyle=\ttfamily,
  breaklines=true,
  numbers=left,
  numberstyle=\tiny,
  stepnumber=1
}

\lstdefinestyle{zlua}{
  language=[5.2]Lua,
  % TJN: If have problems with Lua language for listings package, comment
  % above and uncomment the following line to use 'C' as language type.
  %language=C,
  showspaces=false,
  columns=fullflexible,
  breaklines=true,
  frame=single,
  commentstyle=\rmfamily,
  emphstyle=\bf,
  basicstyle=\footnotesize,
  stringstyle=\ttfamily,
  breaklines=true,
  numbers=left,
  numberstyle=\tiny,
  stepnumber=1
}

\lstdefinestyle{my-bourne-numbered}{
  language=bash,
  showspaces=false,
  columns=fullflexible,
  breaklines=true,
  frame=single,
  emphstyle=\bf,
  basicstyle=\footnotesize\ttfamily\scriptsize,
  commentstyle=\itshape\color{mygray},
  stringstyle=\ttfamily,
  breaklines=true,
  numbers=left,
  numberstyle=\tiny,
  stepnumber=1
}

\lstdefinestyle{my-docker-numbered}{
  language=bash,
  showspaces=false,
  columns=fullflexible,
  breaklines=true,
  frame=single,
  emphstyle=\bf,
  basicstyle=\footnotesize\ttfamily\scriptsize,
  commentstyle=\itshape\color{mygray},
  stringstyle=\ttfamily,
  breaklines=true,
  numbers=left,
  numberstyle=\tiny,
  stepnumber=1
}

\lstdefinestyle{my-xml-numbered}{
    language=XML,
    showspaces=false,
    numbers=left,
    numberstyle=\tiny,
    basicstyle=\footnotesize\ttfamily\scriptsize,
    stringstyle=\ttfamily,
    showstringspaces=false,
    keywordstyle=\scriptsize,
    emphstyle={\color{Cyan}},
    morecomment=[s]{<!--}{-->},
    commentstyle=\itshape\color{mygray},
}


%
% PDFpages options
%  \usepackage[<opts>]{pdfpages}
%           final: inserts pages (default)
%           draft: does not insert pages, prints box & the filename instead
%   enable-survey: activates survey functionality (experimental)
%
% Usage:
%   \includepdf[<key=val>]{<filename>}
%        <key=val>: A comma seperated list of options (see package docs)
%       <filename>: Filename of the PDF doc (no blanks!)
%
% See also:
%   http://www.ctan.org/tex-archive/macros/latex/contrib/pdfpages/pdfpages.pdf
%
\usepackage[final]{pdfpages}
%\usepackage[draft]{pdfpages}


%
% Appendix package
%
\usepackage{appendix}



%
%-------PDF stuff--------
%
% see http://ringlord.com/publications/latex-pdf-howto/
% or [Kopka99book, p394-397]
\newif\ifpdf
\ifx\pdfoutput\undefined
	\pdffalse       % we are not running pdflatex
\else
	\pdfoutput=1    % we are running pdflatex
	\pdftrue
\fi
\ifpdf
% You can set the title, author etc. as needed...
\usepackage[pdftex,
	pdftitle={Fault Tolerance and System-level Virtualization in Large-Scale
    Computing},
	pdfauthor={Thomas J. Naughton, III},
	pdfsubject={},
	pdfpagemode={UseOutlines},  
	bookmarks,bookmarksopen,
	pdfstartview={FitH},
    % TJN: To make PDF hyperlinks in paper black, comment/uncomment 'linkcolor'
    % colorlinks,linkcolor={black},citecolor={black},
	colorlinks,linkcolor={blue},citecolor={blue},
	urlcolor={red},
]{hyperref}
\fi
%-------PDF stuff--------



%
%-------DRAFT stuff--------
%
% 'Draft' watermarks
%
% Using options, to lighten the watermark, and only do it on select page/area
%\usepackage[light,outline,first,bottom]{draftcopy}
%\usepackage[light,outline,first]{draftcopy}
%
% TJN: pdfdraftcopy seems to work with both PS & PDF, but w/ pdflatex issues
%\usepackage{pdfdraftcopy}
% so I typically just include the following & ps2pdf the file.
%\usepackage{draftcopy}
%
% - - - - - - - - - - - - - - - - 
% *** Yet Another Method ***
%
% Look at the 'svn-multi' (aka svnkw) page from CTAN, which allows
% for typesetting control over SVN revision information on a per-file
% and/or whole document level.
% See also:
%   - The "LaTeX & Subversion" link/info in the Trac wiki for "pubs"
%      https://bear.csm.ornl.gov/trac/pubs/
%
%  \usepackage{svn}
%  \usepackage{svn-multi}
%    ...
%  \svnid{$Id: defs.tex 3825 2015-06-26 05:44:36Z tjn3 $}
%    ...
%  Paper Revision:~\svnrev \\
%

\usepackage{svn}
%\usepackage{svn-multi}
\usepackage{svnkw}

%-------DRAFT stuff--------


%
% This gives the heading the funny little footnote markers...then switch to 
% standard arabic as shown below for main body of article.  
% Ref: [p.110-111]{Kopka99book}
\renewcommand{\thefootnote}{\fnsymbol{footnote}}




%-----------------------------------------
% Header / Footer stuffo
%
% NOTE: Using the '\draftmark' macro, which must
%       be defined prior to use of this code.
%       This is typically done in the top-level
%       LaTeX document that is including this 
%       'defs.tex' file.

\usepackage{fancyhdr}
  \pagestyle{fancy}
  \fancyhf{}                             % Clear all other settings
  \renewcommand{\headrulewidth}{0.0pt}   % get rid of the rule-line
  \lhead{}
  \chead{} 
  \rhead{}
  \lfoot{}
  \cfoot{}
  \lfoot{\begin{small}\emph{\draftmark}\end{small}}

%-----------------------------------------



%%%%
% (TJN: Used for \xxxtodo and \xxxtodoMargin macros)
%
% textwidth sets width of todo items in margin
\usepackage[colorinlistoftodos, textwidth=2.5cm, shadow]{todonotes}
%
% Uncomment to DISABLE all todonotes
%\usepackage[disable, colorinlistoftodos, textwidth=2.5cm, shadow]{todonotes}
%%%%

